\documentclass{article}

\usepackage[margin=1in]{geometry}

\usepackage{graphicx}
\begin{document}

		\title{A REPORT ABOUT TREE SPECIES IN MAKERERE UNIVERSITY}
		\author{Author : TAYEBWA HAROLD  }
		
                      \date{Reg no:15/U/13119/PS }
		\maketitle
	

	\tableofcontents
\section{acknowledgement}
  I would like to extend my gratitude thanks to the management of Makerere University which allowed me to make a survey and report about their environment.
I also thank students of this mighty university for sharing their views with me.
 Special thanks also go to my lecturer Mr. Earnest Mwebaze for the brilliant idea he gave me and of course not forgetting my parents for the financial support.

\section{Abstruct}

    Makerere University is one of the best institutions of learning in Africa. It being among the best it takes a close looks at her environment.
This is done by planting different tree species, grass and flowers, but this research is mainly about the tree species planted.

\section{Introduction}

    Makerere University was established in 1921, and during those early days of its start, it had a poor environment setting and according to prof.Behayo Hillary, this was due to low funds and inadequate knowledge about environment conservation by then.
But it is evident that some of these challenges have been swayed away with time.

\subsection{methodology}
    
    Different research methods were used to conduct a research on this topic.
These included interviewing some students and lecturers, filling questioners on the environment, using cameras to take images of the compound and trees, also used internet to make a research on the background state of this university’s compound 
    
\section{Summary of results}
Different tree species are planted with in around Makerere University for different purposes
  Some tree species like palm tree which are planted along the college of natural sciences were planted to beautify the place, arcadia trees planted along the main roads in the university, were planted to provide shade to students.
Some trees like mahoganies that are planted along the main gate can be used for timbre and fire wood when they grow enough.
According to the Makerere University minister of estates and environment, Mr. Ndyomugabe Owen, trees species planted with in the university are of different prices, which range from ten thousand Ugandan shillings to twenty five Ugandan shillings.

The screen shots below are of the form that will be filled in with the necessary requirements
\graphicspath{ {images/} }

\includegraphics[width=5cm, height=8cm]{name}
\graphicspath{ {images/} }

\includegraphics[width=5cm, height=8cm]{first}

\graphicspath{ {images/} }

\includegraphics[width=5cm, height=8cm]{second}

\graphicspath{ {images/} }

\includegraphics[width=5cm, height=8cm]{third}
\graphicspath{ {images/} }

\includegraphics[width=5cm, height=8cm]{fourth}

\section{Conclusion}
  Makerere University has done a lot to conserve and preserve her environment and this is evidenced by its green environment and beautiful trees.

\end{document}